\input hhletter                     % format for 8.5 by 11 paper
%\input hhsmall                      % format for viewing on phones and tablets

\usepackage{hyperref}              % causes yap to not preview .dvi files
%\newcommand{\url}[1]{{\tt #1}}      % clumsy workaround

\begin{document}

\renewcommand{\baselinestretch}{1.1}

\begin{center} {\sc
R and RStudio tutorial - Math 4450/5450 - Spring 2015 - Zirbel}\\
\end{center}

\renewcommand{\blist}[1]{\begin{list}{{\arabic{enumi}:}}{\usecounter{enumi}\setlength{\itemsep}{#1}}}
\renewcommand{\elist}{\end{list}}

R is a programming language that is used for statistics.
RStudio is an Integrated Development Environment (IDE), a place where you can edit R programs, run them, and keep track of the variables in the programs and the graphics windows that R produces.

\minorheading{Installing R and RStudio}

If you installed R a long time ago, you might want to uninstall it and install the most recent version.  I just installed version 3.1.2.

To install R, go to \url{http://cran.rstudio.com/}, find your platform, and pick your way through the links that get R installed.  Generally, you would rather use compiled binaries than source code, tarballs, or anything like that.

To install RStudio, go here:  \url{http://www.rstudio.com/products/rstudio/download/} and look for the section called {\bf Download RStudio Desktop v0.98.1091} or some higher-numbered version.
I suggest that you find the Installer for your platform, click that, wait patiently, and install it.

When you start RStudio, the window will be divided into sub-windows.
One of these is the Console window, with the prompt $>$.
The commands below should be typed at the prompt.
Don't type the line numbers or the comments at the ends of the lines.

\minorheading{R basics}
\blist{-0.05in}
\vspace{-0.1in}
\item {\tt 2*9}       \hfill R can be used as a calculator.
\item {\tt exp(1)} \hfill The exponential function.
\item {\tt 2\^{}999}
\item {\tt x=exp(1)}  \hfill You can store values with variable names.
\new This sets the value of {\tt x} equal to {\tt exp(1)}.
\item {\tt x}         \hfill This displays the value of {\tt x}
\item {\tt x <- exp(1)} \hfill Alternate way to assign a value to {\tt x}, often used by people writing about R.  \new But why use two characters when one will do?
\item {\tt x=runif(10)} \hfill {\tt x} is a collection of 10 pseudorandom
numbers, uniformly distributed between 0 and 1. \new {\tt x} is like a row
of cells in a spreadsheet.  It is called a {\bf vector}.
\item {\tt x+5}       \hfill Add 5 to each entry of {\tt x}.
\item {\tt 1000*x}       \hfill Multiply each entry of {\tt x} by 100.
\item {\tt round(150*x)} \hfill Round to the nearest integer.
\item {\tt ceiling(100*x)} \hfill Round up to the next integer. This is how you generate \new random numbers uniformly distributed from 1 to 100.
\item {\tt x=runif(100)} \hfill Go ahead, try something larger than 100.
\item {\tt max(x)} \hfill
\item {\tt mean(x)} \hfill
\item {\tt sum(x)} \hfill
\item {\tt median(x)} \hfill
\item {\tt sort(x)} \hfill Sort x in increasing order.
\item {\tt cumsum(x)} \hfill A vector of cumulative sums of entries of {\tt x}.

\minorheadinginlist{Matrices}
\item {\tt P = matrix(0,ncol=20,nrow=20)} \hfill Examine {\tt P} to see what you get.
\item {\tt rowSums(P)} \hfill Calculate the sums of the rows of {\tt P}.
\item {\tt P[1,2] = 0.5} \hfill Set one value in the matrix {\tt P}.
\item {\tt P[1,1:3] = 1/3} \hfill Set three values in the matrix {\tt P}.
\item {\tt P[10,]} \hfill Display row 10 of the matrix {\tt P}.

\minorheadinginlist{Plotting points and histograms}
\item {\tt plot(x,pch=16)} \hfill Plot the values of {\tt x} against numbers 1, 2, ... as filled dots.
\item {\tt plot(cumsum(x),type="l")} \hfill Plot the cumulative sum against the numbers 1, 2, ..., \new connecting points with lines. Lowercase L.
\item {\tt dev.off()} \hfill Clear the plotting window.
\item {\tt hist(x)} \hfill Make a histogram of the values of {\tt x}.
\item {\tt hist(x,breaks=20)} \hfill Use 20 bins instead of the default 10.
\item {\tt y=-log(x)} \hfill Take the natural logarithm of the entries of
{\tt x}.
\item {\tt hist(y,breaks=20)} \hfill Looks like an exponential distribution.  It will look better \new with more data points, so re-define {\tt x} and do it again.
\item {\tt x=rnorm(10000)} \hfill Normally-distributed pseudorandom numbers.
\item {\tt hist(x,breaks=30)} \hfill A bell-shaped distribution. 
\item {\tt ?hist} \hfill To get help, type ? followed by the name of a command.  Sometimes \new you can guess what you want and then read about it.  Or google your question, like ``R hist''.

\item {\tt up arrow} \hfill Press the up arrow to return to previous commands, hit enter to run them again.
\item {\tt control-L)} \hfill On a PC, clear all the old commands in the console window.
\item {\tt cat("\textbackslash 014")} \hfill On a Mac, clear all the old commands in the console window.

\minorheadinginlist{Downloading R programs for random processes from Github}
\item I have posted a number of R programs on Github.  To find them, google ``Github zirbel random processes'' or go to \url{https://github.com/clzirbel/Random\_Processes} and find the Download Zip button or go straight to \url{https://github.com/clzirbel/Random\_Processes/archive/master.zip} to get the .zip file.
Unzip the file and save it in an accessible place.

\item In RStudio, use the File, Open File menu to navigate to where you unzipped the programs from Github, then find the R folder within it, and open the file {\tt gambler\_outcome.R}.  It will open in an editor window.  It is a good idea to read the commands in it and start to understand what they do.  I have added many comments to the commands in the file, separated from the commands by the \# character.

\item {\tt getwd()} \hfill Figure out what R considers to be the working directory.  This is where it looks for programs and where it writes out files.  I suggest that you use the directory where you have downloaded and unzipped files from Github.  The easiest way to change the working directory is to go to the Session menu, Set Working Directory, To Source File Location if you have already opened a file.  Or, use Choose Directory.

\item {\tt source("gambler\_outcome.R")} \hfill The source function will load the file and run the commands in it.  Make sure you have the working directory set right.  Instead of typing this, you can click the button that says Source in the upper right of the editor window.  The program simulates gambler's wealth for 80 steps.  If the game goes longer than that, it does all the remaining steps at once.  (I could not figure out how to get R to extend the axes dynamically.)

\item Run the program {\tt gambler\_outcome.R} multiple times, then quit.  In the program, find the line that pauses after each step, set the pause time to 0.0, and then source it again.  Now each game will take just a moment to generate and draw, and you can get a better sense of what they look like.

\minorheadinginlist{R functions}
\item In RStudio, use File, Open File to open {\tt source("gambler\_functions.R")}.  Use the Source button on the upper right of the editor window to have RStudio load the file and run the commands in it.
  
\item {\tt P = gambler\_transition\_matrix()} \hfill Set up the transition matrix for Gambler's wealth.
\item {\tt P} \hfill Show the matrix.

\item {\tt P = gambler\_transition\_matrix(10,20,0.5)} \hfill Specifically tell R what initial wealth, opponent wealth, \new and win probability to use.  Check that {\tt P} looks the same.

\item {\tt P = gambler\_transition\_matrix(5,5,0.7)} \hfill Set up a transition matrix for a \new different gambling situation.  Look at {\tt P} now.

\item Read through the file {\tt gambler\_functions.R}.  There are several function definitions, for example, this one:  {\tt gambler\_transition\_matrix = function(m=10,n=20,p=0.5)}.  It says that there are three inputs, which will be called {\tt m, n,} and {\tt p} within the function.  If the user does not give values for them, the default values will be 10, 20, and 0.5, respectively.

\minorheadinginlist{Matrix operations and plotting}

\item {\tt install.packages("expm")} \hfill Do this once to install the package on your computer.
\item {\tt library(expm)} \hfill Do this once per R session to use the expm library, which allows R to calculate \new matrix powers.  If you source matrix\_functions.R (see below), then this is done for you.
\item {\tt install.packages("gplots")} \hfill Do this once to install the package on your computer.
\item {\tt source("matrix\_functions.R")} \hfill Load functions for matrix calculations.

\item {\tt print\_matrix(P)} \hfill Show the matrix {\tt P} so that it can be copied and pasted into another program easily.

\item {\tt transition\_matrix\_powers(P,0)} \hfill Show six powers of {\tt P}.  If you get ``Error in plot.new() : figure \new margins too large'' try making the figure window larger.

\item {\tt A = P \%\^{}\% 20} \hfill Calculate the 20th power of {\tt P} and store as a new matrix {\tt A}.  This uses the expm package.

\item {\tt pdf("filename.pdf",width=8.5,height=11)} \hfill If you want to save a plot as a PDF, type this command, \new then make the plot, and then do the next command to actually write the file.
\item {\tt dev.off()} \hfill Do this after plotting, to save the file.

\end{list}
\vfill
\end{document}
