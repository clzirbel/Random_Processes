%\input hhletter                     % format for 8.5 by 11 paper
\input hhsmall                      % format for viewing on phones and tablets

\usepackage{hyperref}              % causes yap to not preview .dvi files
%\newcommand{\url}[1]{{\tt #1}}      % clumsy workaround

\begin{document}

\renewcommand{\baselinestretch}{1.1}

\begin{center} {\sc
Matlab tutorial - Math 4450/5450 - Spring 2015 - Zirbel}\\
\end{center}

\renewcommand{\blist}[1]{\begin{list}{{\arabic{enumi}:}}{\usecounter{enumi}\setlength{\itemsep}{#1}}}
\renewcommand{\elist}{\end{list}}

Matlab is available in computer labs on the BGSU campus.
To find which ones, go to 
When you start Matlab, the {\bf command window} appears, with $\gg$ as the prompt.
The commands below should be typed at the prompt.
Don't type the line numbers or the comments at the ends of the lines.
If you are using Octave

\minorheading{Matlab basics}
\blist{-0.05in}
\item {\tt 2*9}       \hfill Matlab can be used as a calculator.
\item {\tt sin(1)}
\item {\tt 2\^{}999}
\item {\tt x=sin(1)}  \hfill You can store values with variable names.
\item {\tt x}         \hfill This displays the value of {\tt x}
\item {\tt x=rand(10,1)} \hfill {\tt x} is a collection of 10 random
numbers, uniformly distributed between 0 and 1.  {\tt x} is like a column
of cells in a spreadsheet.  It is called a {\bf vector} or a 10 by 1 {\bf
matrix}.
\item {\tt x+5}       \hfill Add 5 to each entry of {\tt x}.
\item {\tt 100*x}       \hfill Multiply each entry of {\tt x} by 100.
\item {\tt ceil(100*x)} \hfill Round up to the next integer. This is how you generate \new random numbers uniformly distributed from 1 to 100.
\item {\tt x=rand(10,7)} \hfill A 10 by 7 matrix of random numbers.  To repeat this, or any other command, \new hit the up arrow as many times as you want.
\item {\tt x=rand(100,1)} \hfill Go ahead, try something larger than 100.
\item {\tt x=rand(100,1);} \hfill Use a semicolon to suppress the output.
\item {\tt max(x)} \hfill
\item {\tt mean(x)} \hfill
\item {\tt sum(x)} \hfill
\item {\tt median(x)} \hfill
\item {\tt cumsum(x)} \hfill A vector of cumulative sums of entries of {\tt x}.
\item {\tt y=sort(rand(10,1))} \hfill Sort another 10 random numbers.

\minorheadinginlist{Matrices}
\item {\tt P = zeros(20,20)} \hfill Examine {\tt P} to see what you get.
\item {\tt sum(P)} \hfill Calculate the sums of the columns of {\tt P}.
\item {\tt sum(P')} \hfill Transpose first to calculate the sums of the rows of {\tt P}.
\item {\tt P(1,2) = 0.5} \hfill Set one value in the matrix {\tt P}.
\item {\tt P(1,1:3) = 1/3} \hfill Set three values in the matrix {\tt P}.
\item {\tt P(10,:)} \hfill Display row 10 of the matrix {\tt P}.

\minorheadinginlist{Plotting points and histograms}
\item {\tt plot(x)} \hfill Plot the values of {\tt x} against numbers 1, 2, ...
\item {\tt hist(x)} \hfill Make a histogram of the values of {\tt x}.
\item {\tt hist(x,30)} \hfill Use 30 bins instead of the default 10.
\item {\tt y=-log(x)} \hfill Take the natural logarithm of the entries of
{\tt x}.
\item {\tt hist(y,30)} \hfill Most of the values of {\tt y} are near 0, but
some are as large as 6.  The numbers in {\tt y} have what is called the
{\bf exponential distribution}.
\item {\tt x=rand(10000,1)} \hfill You typed a command similar to this a
while ago.  Type {\tt x=} and then the up arrow to bring it back, then edit
it to change 100 to 10000.
\item {\tt hist(-log(x),30)} \hfill Looks better now.
\item {\tt z=randn(1000,1);} \hfill Generate 1000 normally distributed numbers
\item {\tt hist(z,30)} \hfill The histogram is bell--shaped.  You might
want to try this with more than 1000 numbers.
\item {\tt help hist} \hfill To get help, type ``help'' followed by the name of a command.  Sometimes \new you can guess what you want and then read about it.  Or google your question, like ``matlab hist''.

\minorheadinginlist{Getting some Matlab programs}
\item I have posted a number of Matlab programs on Github.  To find them, google ``Github zirbel random processes'' or go to \url{https://github.com/clzirbel/Random\_Processes} and find the Download Zip button or go straight to \url{https://github.com/clzirbel/Random\_Processes/archive/master.zip} to get the .zip file.
Unzip the file and save it in an accessible place.

\item In Matlab, use the dropdown above the command window to set the working directory to the Matlab folder in the folder you downloaded from Github.

\item Open the file {\tt gambler\_outcome.m}.  It will open in an editor window.  It is a good idea to read the commands in it and start to understand what they do.  I have added many comments to the commands in the file, separated from the commands by the \% character.

\item {\tt pwd} \hfill Figure out what Matlab considers to be the working directory.  This is where it looks for programs and where it writes out files.  

\item {\tt gambler\_outcome} \hfill Matlab will load the file and run the commands in it.  When you are done running the program, press Control-C to break execution.

\item Run the program {\tt gambler\_outcome} multiple times, then quit.  In the program, find the line that pauses after each step, set the pause time to 0.0, and then run it again.  Now each game will take less time to generate and draw, and you can get a better sense of what they look like.


\end{list}
\vfill
\end{document}
