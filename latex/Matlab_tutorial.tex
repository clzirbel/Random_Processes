\documentstyle[11pt]{report}
\renewcommand{\baselinestretch}{1.1}
\input handout_header

\begin{center} \mbox{\sc
Matlab tutorial - Math 4450/5450 - Spring 2015 - Zirbel}\\
\end{center}

\renewcommand{\blist}[1]{\begin{list}{{\arabic{enumi}:}}{\usecounter{enumi}\setlength{\itemsep}{#1}}}
\renewcommand{\elist}{\end{list}}

When you start Matlab, the {\bf command window} appears, with $\gg$ as the prompt.
The commands below should be typed at the prompt.
Don't type the line numbers or the comments at the ends of the lines.

\vspace*{0.2in}
\hspace*{-0.3in}
{\bf Matlab basics}
\vspace*{-0.07in}
\blist{-0.05in}
\item {\tt 2*9}       \hfill Matlab can be used as a calculator.
\item {\tt sin(1)}
\item {\tt 2\^{}999}
\item {\tt x=sin(1)}  \hfill You can store values with variable names.
\item {\tt x}         \hfill This displays the value of {\tt x}
\item {\tt x=rand(10,1)} \hfill {\tt x} is a collection of 10 random
numbers, uniformly distributed between 0 and 1.  {\tt x} is like a column
of cells in a spreadsheet.  It is called a {\bf vector} or a 10 by 1 {\bf
matrix}.
\item {\tt x+5}       \hfill Add 5 to each entry of {\tt x}.
\item {\tt 1000*x}       \hfill Multiply each entry of {\tt x} by 100.
\item {\tt x=rand(10,7)} \hfill A 10 by 7 matrix of random numbers.  To repeat this, or any other command, hit the up arrow as many times as you want.
\item {\tt x=rand(100,1)} \hfill Go ahead, try something larger than 100.
\item {\tt x=rand(100,1);} \hfill Use a semicolon to suppress the output.
\item {\tt max(x)} \hfill
\item {\tt mean(x)} \hfill
\item {\tt sum(x)} \hfill
\item {\tt median(x)} \hfill
\item {\tt cumsum(x)} \hfill A vector of cumulative sums of entries of {\tt x}.
\item {\tt y=sort(rand(10,1))} \hfill Sort another 10 random numbers.
\item {\tt plot(x)} \hfill Plot the values of {\tt x} against numbers 1, 2, ...
\item {\tt hist(x)} \hfill Make a histogram of the values of {\tt x}.
\item {\tt hist(x,30)} \hfill Use 30 bins instead of the default 10.
\item {\tt y=-log(x)} \hfill Take the natural logarithm of the entries of
{\tt x}.
\item {\tt hist(y,30)} \hfill Most of the values of {\tt y} are near 0, but
some are as large as 6.  The numbers in {\tt y} have what is called the
{\bf exponential distribution}.
\item {\tt x=rand(10000,1)} \hfill You typed a command similar to this a
while ago.  Type {\tt x=} and then the up arrow to bring it back, then edit
it to change 100 to 10000.
\item {\tt z=randn(1000,1);} \hfill Generate 1000 normally distributed numbers
\item {\tt hist(z,30)} \hfill The histogram is bell--shaped.  You might
want to try this with more than 1000 numbers.

\hspace*{-0.5in}
{\bf Getting some Matlab programs}

\end{list}
\vfill
\end{document}
